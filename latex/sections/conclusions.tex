\chapter*{Conclusions}
\label{chap:concl}
\addcontentsline{toc}{chapter}{\nameref{chap:concl}} %manually adding the unnumbered chapter to toc

In this project, I face the problem of synthetic histological image generation for the purpose of training Neural Networks for the segmentation of real histological images. The manual analysis of histological specimen is a complex, time consuming, and expensive task and nevertheless it is a pillar of countless diagnostic techniques. Any form of support for this procedure hence is welcome and endorsed by the health-care system. In particular, in this work, I focus on the problem of histological specimens segmentation. The most advanced algorithms for image segmentation are based on Deep Learning and requires the training of extensive and complex neural networks. One of the toughest hurdles to overcome for the training of those NN is the abundance and the quality of pre-labeled examples of segmentation on real histological samples. The collection of hundreds of hand-labeled histological samples, with pixel-level precision, is virtually impossible. This work thus proposes a methodology to generate, in a completely automatic way, synthetic pre-labeled histological-like images, that can be used as training material for a NN.

The method I propose consists of the recreation of the traditional histological specimens' preparation, and it is based on the sectioning of a 3D virtual model of a region of histological tissue. The virtual 3D model of a region of a particular type of human tissue is built after physical and physiological considerations. The model is then subject to a virtual sectioning operation, which yields the synthetic sampling of the virtual tissue in which the histological identity of every pixel is perfectly known. This first image will act as a segmentation mask for a second, realistic image. In fact, on top of this first image are applied several aesthetical processing and refinements and the final product is the synthetic histological-like image. The pair made of the two images is perfectly suitable for the supervised learning of a NN oriented toward the segmentation of histological images. The production of each pair of images is completely automatic and it does not require the intervention of any human operator, it is thus a scalable process that can produce a great abundance of images. The quality of the images is directly connected to the richness and the quality of the model. The perfect modelization of a region of tissue, let's say human pancreatic tissue, is by far out of reach for this, hence the richness and the fidelity of the produced images are inevitably lower than the real sample. Nevertheless, the quality of the produced images is sufficient to perform the preliminary phase of the training of a NN following a training strategy known as curriculum learning. This learning process consists of giving the NN a copious quantity of lower complexity level example in the first instance, reserving the few and sophisticated real hand-labeled histological samples for the finalization of the training.

The first chapter of this thesis is devoted to the contextualization of the present work. It is offered a description of how the histological samples are obtained after a tissue biopsy and how the digitalization process of the images works.

The second chapter collects all the details of every less common technical tool I used during the design of this project. A brief theoretical introduction is proposed for every item besides the thorough description of its practical use. In this chapter, a section is devoted to the description of a general methodology for computing the 2D section of an arbitrary three-dimensional polyhedron. The algorithm here described has been devised and implemented all by my self, and then inserted in the workflow of the project. This is one of the key pieces for the automatization of the virtual tomography process, and it allows to connect the three dimensional model to the two-dimensional representation of a sampled section.

The third chapter is the center of this work, and it contains the description of all the design choices, and the steps I followed for the development of the two human tissue models I propose: the first of pancreatic tissue and the second of dermal tissue. The first and second sections are dedicated to the description of the two proposed 3D virtual tissue models, which consists of different steps. The third section instead contains a thorough description of the method to perform the sectioning onto a virtual model and how to process the resulting images. The development has required the harmonization of many different technical aspects and mathematical tools and it results in a general methodology for the generation of synthetic histological images. The process passes first through the building of the target tissue's structure in a virtual environment. This structure is then embedded in a three-dimensional space decomposition which subdivide the volume in individual cells. Those cells are labeled in correspondence to their role in the model, and their identity is then perfectly encaptured by the virtual tomography
procedure. The sectioning process is responsable for the production of synthetic images, which are then conveyed toward an aesthetical enrichment pipeline specialized for the particular target tissue. The product of any application of this process is a pair made of segmentation mask and the correpsonding synthetic histological-like image. This completely automatized procedure allow to built arbitrary large dataset for the training of NN, without the intervention of any human operator.

The method for the generation of datasets of synthetic images I propose in my thesis work is a self-supporting project and it is formally consistent. By the way, there are many possibilities for improvement and enrichment for the project. One first aspect to strengthen could be the richness of the models: adding more elements in the structure, and refining their representation of the tissue at the cellular level. This would lead to a better quality of the synthetic images, that would assist the training of NN in more and different applications. Another aspect that lends itself to improvements in the development of a dedicated style transfer NN targeting the histological texture transfer, which could lead to interesting signs of progress in the quality of image generation. There is also the intention to perform an actual attempt of NN training on the images produced with this process. This would complete conceptually the idea underneath the project and would be an excellent opportunity to detect weaknesses and to draw up possible lines of development. The repeated application of the generation method would allow the building of entire datasets suitable for the training of DL-based models.
