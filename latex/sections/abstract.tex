\chapter*{Abstract}
The project is inspired by an actual problem of timing end accessibility in the analysis of histological samples in the health-care system. In this project, I face the problem of synthetic histological image generation for the purpose of training Neural Networks for the segmentation of real histological images. The method I propose is based on the replication of the traditional specimen preparation technique in a virtual environment. The first step is the creation of a 3D virtual model of a region of the target human tissue. The model should encapture all the key features of the tissue, and the richer it is the better will be the yielded result. The second step is to perform a sampling of the model through a virtual sectioning process, which produces a first image of the section, which will act as a segmentation mask. This image is then processed with different tools to achieve a histological-like aspect. The most significant contribution is given by the action of a style transfer neural network that implants the typical visual texture of a histological sample onto the synthetic image. This procedure is presented in detail for two specific models of human tissue: one of pancreatic tissue and one of dermal tissue. The two resulting images compose a pair of images and corresponding ground-truth image, which is perfectly suitable for a supervised learning technique. The generation process is completely automatized and does not require the intervention of any human operator, hence it can be used to produce arbitrary large datasets. The synthetic images are inevitably less complex than the real samples and they offer an easier segmentation task to solve for the NN. However, the synthetic images are very abundant, and the training of a NN can take advantage of this feature, following the so-called curriculum learning strategy.
